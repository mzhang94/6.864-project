\documentclass[dvips,12pt]{article}

% Any percent sign marks a comment to the end of the line

% Every latex document starts with a documentclass declaration like this
% The option dvips allows for graphics, 12pt is the font size, and article
%   is the style

\usepackage[pdftex]{graphicx}
\usepackage{amsmath, amssymb}
\usepackage{setspace}
\usepackage{float}
\usepackage{url}

% These are additional packages for "pdflatex", graphics, and to include
% hyperlinks inside a document.

\setlength{\oddsidemargin}{0.25in}
\setlength{\textwidth}{6.5in}
\setlength{\topmargin}{0in}
\setlength{\textheight}{8.5in}
\setlength{\parindent}{0pt}

% These force using more of the margins that is the default style

\begin{document}


\title{Learning Character Graphs from Literature}
\author{Sumit Gogia, Min Zhang, Tommy Zhang}
\date{\today}

\maketitle

\section{Introduction}
    Literary scholars, when comparing different works of literature, frequently use character roles and relationships to make their comparisons. These roles include major single-character literary labels such as ``protagonist'', as well as two-character relationship labels such as ``parent-child''. These character relationships for a novel can be visualized as a graph, an example of which is shown below: \\

    [insert graph example] \\
    %% character relationship graph example

    We propose, for this project, to build a system that automatically produces these graphs, which we call \emph{character graphs}, given the text for a work of literature. While literary scholars frequently hypothesize trends in literature by performing a small scale analysis of different works manageable through close reading and manual annotation, having a system that can automatically visualize these character relationships can help with exploring trends across a large number of novels.

\section{Methods}
\subsection{Problem Formulation}
    To be more precise about the problem, for each novel, we will extract a list of major characters in the novel. We will have some
    predefined labels to describe roles of characters, like ``protagonist'' and ``antagonist'', and some predefined labels to describe relationships between two characters,
    like ``father-daughter'' or ``love-love''. The relationship labels could be a pair of nouns or verbs. This way, we can view character extraction as a name entity recognition (NER) problem,
    and character roles and relationships extraction as a classification problem.

\subsection{Data Collection}
    We found no well formatted data that labels character roles and relationships in a way that we can directly use as annotated training data.
    To avoid manual annotation, we plan to write a simple extrator that automatically extracts character roles and relationships from
    descriptions of fictions on Sparknote. We expect that this will not be difficult to do because the information on Sparknote is very
    well formatted and concise. For each novel, Sparknote provides a list of major characters and a short description for each major
    character. Character roles and relationships are usually summarized in the first few sentences of the description. For example,
    for the novel ``The Great Gatsby'', the description of character Jay Gatsby begins with ``Jay Gatsby -  The title character and protagonist of the novel''
    and the description of Daisy Buchanan begins with ``Daisy Buchanan - Nick's cousin, and the woman Gatsby loves''.

    The list of major characters can be obtained directly by parsing the html of the webpage. We restrict roles of characters to a fixed set which contains
    protagonist, antagonist, sidekick, tempter, etc. We will simply look for these words in character descriptions. To extract the relationships between characters,
    we will look for sentences where names of both characters appear and extract the verb or noun of that sentence. We will then manually look at the words we extract
    and pick some to form the set of labels to describe relationships between characters.
\subsection{Baseline}

\subsection{Our Approach}
    Our proposed system takes the form of a pipeline of subsystems that extract character references from the text, determine semantic representations of them, and then classify their literary roles and relationships from this semantic representation. The pipeline is visualized below: \\

    [Insert Pipeline Graphic] \\

    The neural networks we train, for extracting semantic character representations (\emph{character vectors}) and classifying literary relationships, are also visualized separately for clarity: \\

    [Begin figure of neural networks] \\

    Each of these networks is trained across the entire dataset of novels obtained from data collection, though we train the character vector network before the other two, as they require semantic character representations as input.  \\

    Our hypothesis is that a semantic character vector can be obtained for each character given context for the character's appearance in a novel, using the first network. The exact context that we should use is unclear - however, for a start, we believe that using the dependent verbs and direct objects in sentences of character appearance will be helpful. If this proves difficult to work with, either due to issues with training to target a high-dimensional sparse output, or beause we are unable to extract these features, we can fall back to features used in [insert McKeowan citation] for their manual method. \\

    Given output from the first network, the other two networks are similar to many current classification networks using word vectors as input. We feed in a singlet or pair of characters, and then obtain an output label corresponding to their literary role or relationship. To keep learning feasible, we restrict to a small subset of major literary role and prominent relationship labels. Since the training is done over all novels, we believe we will have enough data to perform adequate learning. \\

    After we have passed the characters through the last two networks, we then have labels for each character and a list of labeled 2-character relationships. As described in the introduction, these labels then induce a graph structure that we can visualize.

\subsection{Result Measuring}
\section{Milestones}
    We plan to finish data collection before November 14, 2015, baseline before November 21, implementation of our approach before December 1, final report before December 11.

\end{document}
